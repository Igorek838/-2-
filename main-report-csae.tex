\documentclass[14pt, oneside]{altsu-report}

\worktype{Курсовая работа (2 курс)}
\title{Морской бой на C++}
\author{И.\,О.~Шелепов}
\groupnumber{5.205-2}
\GradebookNumber{1337}
\supervisor{И.\,А.~Шмаков}
\supervisordegree{Старший преподаватель}
\ministry{Министерство науки и высшего образования}
\country{Российской Федерации}
\fulluniversityname{ФГБОУ ВО Алтайский государственный университет}
\institute{Институт цифровых технологий, электроники и физики}
\department{Кафедра вычислительной техники и электроники}
\departmentchief{В.\,В.~Пашнев}
\departmentchiefdegree{к.ф.-м.н., доцент}
\shortdepartment{ВТиЭ}
\abstractRU{Пока счётчик работает не правильно! Поправьте количество рисунков и таблиц в cls-файле.}
\keysRU{Курсовая работа, морской бой, объектно-ориентированное программирование}
\keysEN{computer simulation, distributed version control}

\date{\the\year}

% Подключение файлов с библиотекой.
\addbibresource{graduation-students.bib}

% Пакет для отладки отступов.
%\usepackage{showframe}

\begin{document}
\maketitle

\setcounter{page}{2}
\makeabstract
\tableofcontents

\chapter*{Введение}
\phantomsection\addcontentsline{toc}{chapter}{ВВЕДЕНИЕ}

\textbf{Актуальность}
\newline
Актуальность данной работы заключается в изучении объектно-ориентированного программирования с явным выделением состояний для написания игры "Морской бой". Парадигма объектно-ориентированного программирования используется повсеместно, поэтому её изучение поможет решать какие-либо нужные задачи.
\newline
\textbf{Цель}
\newline
Написать игру "Морской бой" на C++, в которой будет расширенный режим, режим игры с ботом и режим игры 1 на 1 с другим человеком.
\newline
\textbf{Задачи:}
\begin{enumerate}
\item Изучить правила игры "Морской бой";
\item Изучить и освоить парадигму объекто-ориентированного прогрммирования;
\item Изучить и освоить графическую библиотеку Dear ImGui;
\item Написать "Морской бой" на C++.
\end{enumerate}

% Подключение первой главы (теория):
\include{chapter-1-report-csae.tex}
% Подключение второй главы (практическая часть):
\include{chapter-2-report-csae.tex}
% Подключение третий главы (практическая часть с тестированием:
\include{chapter-3-report-csae.tex}

\chapter{Теоретичекская часть}
В этом разделе мы разберём суть работы, условия и задачи. Также изучим необходимые компоненты и теорию, необходимые для выыполнения работы.
\section{Постановка задачи и условий}
\section{Морской бой}
\section{Язык программирования C++}
\section{Объектно-ориентированное программирование}

\chapter{Практическая часть}
Применяя полученные знания, здесь я пошагово распишу этапы разработки игры "Морской бой"


\chapter*{Заключение}
\phantomsection\addcontentsline{toc}{chapter}{ЗАКЛЮЧЕНИЕ}

\begin{enumerate}
\item Пример ссылки на электронный источник~\cite{wikiC++, wikiRUООП, wikiRUImGui, wikiSeaBattle, wikiRUGeany}.
\item Пример ссылки на книгу одного автора~\cite{book1author}.
\item Пример ссылки на книгу 5-ти и более авторов~\cite{book5author}.
\end{enumerate}

\newpage
\phantomsection\addcontentsline{toc}{chapter}{СПИСОК ИСПОЛЬЗОВАННОЙ ЛИТЕРАТУРЫ}
\printbibliography[title={Список использованной литературы}]

\appendix
\newpage
\chapter*{\raggedleft\label{appendix1}Приложение}
\phantomsection\addcontentsline{toc}{chapter}{ПРИЛОЖЕНИЕ}
%\section*{\centering\label{code:appendix}Текст программы}

\begin{center}
\label{code:appendix}Код программы "Морской бой"
\end{center}

\end{document}

